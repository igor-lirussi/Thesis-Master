%----------------------------------------------------------------------------------------
\chapter{Conclusions}
\label{chap:conclusions}
%----------------------------------------------------------------------------------------

This thesis provides a possible approach to novel high-level skill generation by combining movement primitives learned by CNMP models. 

The main key findings of the study are two methods developed to synthesize new actions from demonstrated ones. 

In the first approach, parts of trajectories are blended thanks to the combination of the task interpolation ability of the neural network analyzed and the mathematical system developed to pass the proper parameters to it.  

Furthermore, two different architectural changes have been proposed to the classic CNMP model. The two different architectures achieve similar results compared to the original model, but they enable its use with partial information. 

The second approach presented achieved action synthesis thanks to the concatenation of primitives combined with the spatial interpolation of the network and the ability to encode multidimensional data.

The approaches are not free from limitations, the main ones discovered are the need for an initial condition and the awareness of the environment.  

Finally, the practical applications have been investigated, and both methods proposed performed well on the tests on real-life robots. 

\section{Future work}
Possible future works are multiple since the research touched extensively many topics. 
This research leaves many challenges open to tackle, from planning to a better world representation, perception, and grasping.

Another area in which the network output can be used is navigation. The trajectories analyzed in this study focused on robotics manipulation but can also be extended to robotic navigation.

A possible research in the CNMP with the task parameters only in query would be to condition it with one state without parameters and let the network, queried with different tasks, pass through that state.

Further research is needed on the same network developed for better training to improve the results in the interpolation dimensions. For example, conditioning it with task parameters between the original ones during training. This will force mixed task parameters trajectories to pass through the designated point. 

Moreover, it's possible to extend the research to use the task parameter changing capabilities to shift to another meaningful plan if one fails. The same abilities could be used to change among continuous actions.

In the second part, some further optimization of the graph building can be achieved. More intelligent methods of research and pruning could be developed. 

Furthermore, actions of different time lengths in the concatenation procedure can be integrated. 

Finally, an excellent addition would be extending the simple low-dimensional world representation given to the network in the second method with an environment image. 
