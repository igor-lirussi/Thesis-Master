%----------------------------------------------------------------------------------------
\chapter{Introduction}
\label{chap:introduction}
%----------------------------------------------------------------------------------------

\section{Overview}
Humans have been building sequences of actions to achieve 
decision-making under uncertainty, which is essential for achieving high-level goals

In this study, we propose a computational model that is biologically inspired. Our approach 
Implement it on an anthropomorphic robot and on an industrial collaborative robot.
Specifically, ... 

\section{Challanges}
Robotics dominates many fields, but often the environment is controlled, designed to help the robot in it's task and not human friendly. All environments in which humans are present are nor organized nor predictable 
Some jobs have been substituted by machines but are specific and repetitive. Closest contact consumer public with automation is with machines that have a single skill. Vacuum robot can only vacuum, and it's its only capability, dishwasher can only wash dishes, dough mixer does only
\section{Objectives}
 combine not only movement one after the other (end-to-end) but combine parts of them

\section{Thesis Structure} 
Accordingly, the remainder of this thesis is structured as follows.
\Cref{chap:background} discusses the background of the topic, the current advancements in the field, and the related research with a literature review. 
In \cref{chap:platforms} the instruments and frameworks used in this research are listed and analyzed to be able to understand the initial setup and replicate the results. 
The \cref{chap:design} explains the design and architecture of the proposed method. In order to understand the logic, the conceptual passages and mathematical background. 
The \cref{chap:implementation} analyzes the key points of the implemented solution through the explanation of the most important passages in the code developed.
The \cref{chap:validation} shows the final results and the testing on real-life robotic platforms.
Finally, \Cref{chap:conclusions} concludes this thesis by summarising its main contribution and future work.